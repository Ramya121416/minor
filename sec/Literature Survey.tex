\newpage
%\pagenumbering{arabic}
\begin{center}
    \textbf{\LARGE CHAPTER - 2}
\end{center}
\section{LITERATURE SURVEY}
 The human body is composed of a variety of cell
types. The brain is a delicate and highly specialised organ
of the human body. The benign or malignant form of brain
tumour can be identified. The malignant is cancerous whereas
the benign is not. Malignant tumor is classified into two types:
primary and secondary tumor. Detection of brain tumor is an
one of the challenge for human because of its structure.
Tumors are divided into primary and secondary categories.
Because of its structure, detecting brain tumours is one of the
most difficult tasks for humans.
Brain picture segmentation is necessary for brain tumour
detection. Segmenting MR images of the brain manually is
challenging. It takes a lot of time, is a difficult process
that cannot be repeated, needs non-uniform segmentation, and
segmentation outcomes might differ from expert to expert.
Automated brain tumour segmentation and identification is
fraught with problems. Brain tumour segmentation is a challenging problem for an autonomous computer system since it
requires pathology, MRI physics, as well as intensity and form
analysis of the MRI picture. The main problem in segmenting
brain tumours is that they differ in terms of form, size,
location, and picture intensities. A computer-aided approach
for brain tumour identification and segmentation is preferable
since manual segmentation of brain tumours involves human
specialists and takes a lot of time.
Therefore in this situation, computer assisted system is beneficial. The classification of the brain MR image as normal
or tumorous should be accurate and take less time with an
automated brain tumour detection method. It must be reliable
and provide radiologists an intuitive, user-friendly solution that
is self-explanatory
The most fatal illness, brain tumours have an extremely low
life expectancy in their most severe form. Brain tumour misdiagnosis will lead to ineffective medical treatment and lower
patient survival rates. Developing an appropriate treatment
strategy to treat and prolong the lives of people with brain
tumours depends on making an accurate diagnosis of the
disease.
Using clinical brain scans, Balakumar et al. [2] developed
an automated segmentation of brain tumour identification.
Compared to manual tumour analysis, automatic segmentation
of brain tumours is far superior. In this study, the brain tumour
areas are to be retrieved using region-based segmentation. For
additional processing, the shape, intensity, and texture-based
characteristics need to be extracted. The post processing technique may be used to do the spatial regularisation. The strength
of this research is in its ability to categorise structural elements
into normal and pathological tissue using discriminative traits,
which further reduces complexity. The paper’s flaw is that the
edge-based approach fails when the image is too complicated
or blurry to clearly show a boundary.
A probabilistic neural network-based method for identifying
brain tumours was proposed by Dina Aboul Dahab et al. In this
study, a customised PNN model is built utilising MRI scans
to analyse data and images while learning vector quantization.
At the preprocessing stage, the image’s edges are smoothed
using a Gaussian filter. The color-based segmentation approach
should be utilised in this situation since it produces better
results than the kmeans algorithm. ROI uses a modified version
of the canny edge-based detection technique. The postprocessing stage makes use of the modified PNN algorithm. This
paper’s benefit is that it uses a PNN system based on LVQ,
which lengthens processing time by around 79 percentage of
CNN in PNN.
In recent years, it has been possible to identify brain tumours
in MRI scans using machine learning techniques, most specifically CNN. Yet, the idea of employing MRIs dates to 1984.
For many years, MRI has been a vital imaging tool in the
detection and management of brain malignancies. Manual MRI
interpretation, however, can be laborious, error-prone, and
vulnerable to inter-observer variability. As a result, computeraided diagnosis (CAD) systems have been created to help
radiologists analyse medical pictures.
Convolutional neural networks (CNNs) with transfer learning
have been used to identify brain cancers in MRIs as early
as 2010. Deep learning was becoming more popular at the
time, and scientists were investigating its potential uses in the
processing of medical images. CNN makes it easier to identify
the tumour and takes less time
Havaei et al. carried out one of the first investigations in this
area in 2014. Using T1- weighted contrast-enhanced MRIs,
they classified brain cancers into four groups using a CNN
(glioma, meningioma, pituitary adenoma, and healthy brain
tissue). The network’s accuracy was 90.5being trained on a
dataset of 220 patients.
Another team of University of Michigan researchers
employed transfer learning in 2016 to create a CNN-based
technique for automatically classifying brain tumours in MRI
images. The technique employed a pre-trained CNN model,
which was then refined using a collection of brain MRI
images that had tumour locations labelled. The model was
quite accurate in differentiating between various kinds of
brain cancers. With MRI images, they employed a CNN with
transfer learning to find brain cancers. To extract features from
MRI scans, they employed a pre-trained model (VGG16),
which was initially trained on the ImageNet dataset. They
subsequently improved the model using a dataset of brain
tumour MRIs, and 97 percentage accuracy was attained
\begin{table}[H]
    \centering
    \caption{\small analysis of differents methods.}
   \scriptsize
    \begin{tabular}{lrrrrrrrrrr}
\toprule
{} &  \small Name  &  \small accuarcy   \\

\midrule
1 &    InceptionNet  &    85 percentage   \\
2 &    resNet50  &    89 percentage  \\
3 &    vgg16  &   95 percentage\\

\bottomrule
\end{tabular}
    
    \label{fig:posterior}
\end{table}