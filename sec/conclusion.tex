\newpage

   \begin{center}
    \textbf{\LARGE CHAPTER - 8}
\end{center}
    
\section{CONCLUSION}
Detection of Brain tumor using convolutional neural network by transfer learning using vgg16 . By automating various processes :
\begin{itemize}
    \item High Accuracy
    \item Early Detection
\end{itemize}
 
 the detection of brain tumors using Convolutional Neural Networks (CNNs) and transfer learning, particularly with the VGG16 architecture, has shown great promise and has several noteworthy implications:
High Accuracy: Transfer learning with VGG16 allows the model to leverage pre-trained weights from a large dataset (e.g., ImageNet), enabling it to extract relevant features effectively. This results in a high level of accuracy in brain tumor detection.


Reduced Data Requirements: Transfer learning reduces the need for an extensive dataset specific to brain tumor images. This makes it possible to develop accurate models even with limited medical imaging data, which can be invaluable in healthcare settings where data is often scarce.


Early Detection: Early detection of brain tumors is critical for timely treatment and improved patient outcomes. CNNs with transfer learning can help in identifying tumors at an early stage, potentially saving lives.