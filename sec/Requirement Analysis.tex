\newpage
\pagenumbering{arabic}
\begin{center}
    \textbf {\LARGE CHAPTER - 3}
\end{center}
\section{REQUIREMENT ANALYSIS \& SPECIFICATION}

This section of the project report is the most critical element as, it provides a foundation for the entire project. It ensures that all stakeholders are aligned on the project's objectives, scope, and deliverables, and it provides a clear road map for the project team to follow.\\This section is further divided into 3 sister sections:
\begin{itemize}
    \item Feasibility Study
    \item Model Selection
    \item SRS
\end{itemize}
% \clearpage

\subsection{Feasibility Study}
 A feasibility study is the first stepping stone into the development of any project, including our Tri-
level segmented CNN to predict dental informatics. It involves assessing the potential for the project
to be successful, which in turn includes evaluating the market, technology, financial aspects, and
operational requirements. Conducting a feasibility study on detecting brain tumors in MRIs using a Convolutional Neural Network (CNN) involves assessing the technical, operational, economic, and scheduling aspects of implementing this approach
 
\subsubsection{Market Analysis}
A market analysis involves assessing the current market landscape, potential demand, competition, pricing strategies, and overall market viability. It Provide's an overview of the medical imaging market and the increasing importance of advanced technologies for brain tumor detection and Introduce's the concept of using CNNs with transfer learning for brain tumor detection in MRI images. It Identifies key segments within the market, such as MRI devices, software solutions, and AI-based diagnostic tools
\subsubsection{Technology Assessment}
It involves evaluating the key technologies, tools, advantages, limitations, and future prospects of the approach. It Explain's the fundamentals of CNNs, including their architecture, layers, and operations like convolution, pooling, and fully connected layers. And Describe's how CNNs are particularly well-suited for image processing tasks like medical image analysis. It Discuss the specific ways how  CNNs are utilized in brain tumor detection from MRI images, emphasizing feature extraction and classification
\subsubsection{Operational Requirements}
 Establishing the operational requirements  involves specifying the technical, infrastructure, personnel, and workflow aspects necessary to effectively implement and manage this technology. and Specifies the required computing power, including CPU, GPU, or specialized hardware accelerators, based on the complexity of CNN models and the size of the dataset. It Define's the amount of storage needed for storing MRI datasets, pre-trained models, and intermediate results during training and inference. And also it Determine's the network requirements for transferring large MRI image datasets between storage, preprocessing, training, and inference components.
\subsubsection{Financial Analysis} 
Performing a financial analysis involves assessing the costs associated with developing, deploying, and maintaining the system, as well as potential revenue streams or cost savings generated by its use. Salaries and benefits for data scientists, machine learning engineers, and developers involved in data collection, preprocessing, model selection, training, and integration. And Expenses related are to acquiring, organizing, and preprocessing the MRI data, including any licensing fees, data storage costs, and data augmentation efforts. It Estimate potential revenue or cost savings generated by the brain tumor detection system

\subsubsection{Risk Assessment}
It  involves several risks that should be carefully assessed and managed to ensure the accuracy, safety, and effectiveness of the system. In this the model may be overfit to the training data and perform poorly on unseen data, including misclassifying tumors or generating false positives/negatives. And Implements the  techniques like data augmentation, regularization, and proper model validation to prevent overfitting and enhance model generalization.  The project may exceed the allocated budget due to unforeseen challenges or inadequate cost estimation


\vspace{0.25cm}
\subsection{ Selection of Process Model}
The software life cycle process model is a framework that outlines the various stages involved in the development of a software application. So, choosing a life cycle process model is the stepping stone into the development of a software product.

\subsubsection{ Process Models}
The choice of software development process model for teeth segmentation and future integration of a CNN model for age and gender prediction using GCNN, an iterative and flexible process model would be suitable. One such process model that fits well with AI and deep learning projects is the Agile methodology.

\subsubsection{Why Agile}
Here are some reasons why the Agile model is the best choice for developing teeth segmentation model:

\begin{itemize}
    \item \textbf{Adaptability to Changing Requirements:}  In AI projects, requirements and objectives can evolve rapidly due to advancements in technology or new insights from stakeholders. The Agile model's iterative and incremental approach allows you to adapt to these changes easily, ensuring that your project stays aligned with the latest developments and meets evolving needs.

    \item \textbf{Faster Time-to-Value:} Agile's short development cycles, known as sprints, enable you to deliver working components of the project in a timely manner. This means you can start utilizing and validating the teeth segmentation and AI models early on, providing value to dental professionals and researchers sooner than traditional development approaches.

    \item \textbf{ Continuous Stakeholder Engagement:} Agile emphasizes regular interactions with stakeholders, including dental experts, data scientists, and end-users. This continuous engagement ensures that their feedback and insights are incorporated into the project throughout its lifecycle, resulting in a solution that truly addresses their needs and expectations.
    \item \textbf{Risk Mitigation and Early Issue Identification: } Agile's iterative nature allows for frequent testing and validation of the teeth segmentation and AI models. This early and regular testing helps in identifying potential issues and risks early on in the development process. Addressing these issues promptly minimizes the chances of costly and time-consuming problems later in the project, leading to a more successful and efficient implementation.

\end{itemize}

\subsubsection{Why Not}
Every coin has two sides thus, we can't forget to consider that the waterfall model has some limitations too such as:
\begin{itemize}
    \item Scope Management
    \item Resource Intensive   
\end{itemize}


While Agile offers many benefits, it may not be suitable for projects with fixed, rigid timelines or highly regulated environments, where extensive documentation and upfront planning are mandatory. In such cases, adopting Agile might require careful planning and adaptation to address the specific needs and constraints of the project.













\newpage

\vspace{0.25cm}


\subsection{Software Requirements Specification}
\subsection{Introduction}
\subsubsection{Purpose}
The purpose of detecting brain tumors in MRI images using convolutional neural networks (CNNs) with transfer learning is to improve the accuracy and efficiency of tumor identification. By leveraging pre-trained CNN models, we can harness the knowledge learned from large datasets to enhance the detection of subtle tumor features, aiding in early diagnosis and treatment planning. This approach reduces the need for manual analysis, speeds up the diagnostic process, and ultimately contributes to better patient outcomes in the field of medical imaging.

\subsubsection{Scope}
The Detection of brain tumor with transfer
learning will include the following functionalities:
\begin{itemize}
    \item Enhanced Accuracy: CNNs with transfer learning can provide high accuracy in detecting brain tumors. They can learn relevant features from MRI scans, helping in the differentiation of tumor regions from normal brain tissue. This can aid radiologists in making more accurate and timely diagnoses.
    \item Automation and Efficiency: By automating the initial tumor detection process, CNNs can help radiologists in their workflow, allowing them to focus on more complex tasks and reducing the risk of human error.
\end{itemize}

\subsubsection{Definitions, Acronyms and Abbreviations}
\begin{itemize}
    \item \textbf{Definitions:}\\
    CNN: Convolutional Neural Network\\
    AI: Artificial Intelligence\\
    GCNN: Graph Convolutional Neural Network\\
    SRS: Software Requirements Specification\\
    
    \item \textbf{Acronyms:}\\
    X-ray: Radiograph used in dental imaging\\
    ROI: Region of Interest\\
    FCN: Fully Convolutional Network\\
    IoU: Intersection over Union\\
    GUI: Graphical User Interface\\
    
    \item \textbf{Abbreviations:}\\
    CNN: Convolutional Neural Network\\
    AI: Artificial Intelligence\\
    GCNN: Graph Convolutional Neural Network\\
    SRS: Software Requirements Specification\\
\end{itemize}


\subsubsection{References}
IEEE Std 830-1998, IEEE Recommended Practice for Software Requirements Specifications.

\subsubsection{Overview}
The document will mostly consist of two parts: 
\begin{itemize}
    \item Overall Description
    \item Specific Requirements 
\end{itemize}
Overall description describes the major components of the system, assumptions and dependencies of the system, while specific requirements describes the functions of the system and their roles in the system and the constraints faced by the system.

\subsection{Overall Description}
\subsubsection{Product Perspective}
The product perspective for detecting brain tumors in MRI scans using convolutional neural networks with transfer learning involves implementing advanced machine learning techniques to enhance the accuracy and efficiency of tumor detection. This innovative solution aims to streamline the diagnostic process, providing faster and more reliable results for medical professionals. It leverages pre-trained models to extract meaningful features from MRI images, reducing the need for extensive data annotation and improving generalization. Additionally, the system should offer user-friendly interfaces for radiologists and healthcare practitioners, facilitating seamless integration into clinical workflows. 

\subsubsection{Product Functions}
The key features of the system include:

MRI image upload and processing.
Brain tumor detection using a pre-trained CNN.
User registration and authentication.
User-friendly web interface.


\subsubsection{User Characteristics}
\begin{itemize}
\item \textbf {Radiologists:} These are the primary users who will upload MRI images and review the diagnostic reports.
\item \textbf {Administrators:} Responsible for system maintenance and user management.
\end{itemize}
\subsubsection{Constraints}
The following constraints are considered for the development of the model:
\begin{itemize}
    \item \textbf{Data Availability:} Limited access to diverse and high-quality MRI datasets for brain tumor detection can constrain the effectiveness of convolutional neural networks (CNNs) with transfer learning, as model performance heavily depends on the quantity and diversity of training data.
    \item \textbf{Computational Resources:} Training and fine-tuning CNNs with transfer learning on MRI data often require substantial computational resources, including powerful GPUs and memory, which can be a constraint in resource-limited healthcare settings.
\end{itemize}

\subsubsection{Assumptions \& Dependencies}
The Detecting of brain tumor in MRI’s using a
convolutional neural networks with transfer
learning assumes the following:
\begin{itemize}
    \item  Availability of a sufficient and diverse dataset of MRI brain images containing both tumor and non-tumor cases is essential for training the convolutional neural network (CNN) with transfer learning.
    \item Availability of GPU-accelerated hardware or cloud computing resources is necessary to efficiently train and deploy the CNN
    \item Access to pre-trained CNN models and libraries  for transfer learning is dependent on open-source.
\end{itemize}

\subsection{Specific Requirements}
\subsubsection{External Interfaces}
The Detecting of brain tumor in MRI’s using a
convolutional neural networks with transfer
learning may include the following external interfaces:
\begin{itemize}
    \item Input Data Interface: The system should accept MRI image data as input. These images should adhere to specific formats (e.g., DICOM) commonly used in medical imaging.
    \item User User Interface: If the system is designed for use by medical professionals, create a user-friendly web-based interface or application that allows users to upload MRI images, initiate the detection process, and view the results.
\end{itemize}

\subsubsection{Functions}
The primary functions of the Detecting of brain tumor in MRI’s using a convolutional neural networks with transfer
learning are:
\begin{itemize}
    \item \textbf {Data Preprocessing and Augmentation:}
    \item Data Collection:  Gather a labeled dataset of MRI images with corresponding labels indicating the presence or absence of brain tumors.
    \item Data Preprocessing:  Preprocess the MRI images to enhance features, standardize dimensions, and normalize pixel values.  
    \item \textbf{2.   Transfer Learning:}
    \item Import Pre-trained CNN Model:  Choose a pre-trained CNN model (e.g., VGG16, ResNet, Inception) that was trained on a large dataset (e.g., ImageNet) and remove the final classification layers.
    \item Feature Extraction:  Use the pre-trained CNN to extract features from the MRI images. Freeze the pre-trained layers to retain the learned features.             \item \textbf{Modification and Fine-tuning:}
    \item Add Additional Layers:  Add new layers (fully connected, softmax) on top of the pre-trained CNN to tailor the model for the specific brain tumor detection task.
    \item Fine-tuning:  Unfreeze some of the pre-trained layers and fine-tune the model on the brain tumor dataset to adapt to the specific features relevant to tumor detection.             
    \item \textbf{Compile and Train the Model:}
    \item Compile the Model:  Define the loss function, optimizer, and evaluation metrics (e.g., categorical cross-entropy, Adam optimizer, accuracy).
    \item Train the Model:  Train the modified model using the augmented and preprocessed MRI dataset. Monitor training progress and adjust hyperparameters as needed.            \item \textbf{Model Evaluation and Validation:}
    \item Evaluate the Model:  Assess the model's performance on a separate validation dataset, using metrics like accuracy, precision, recall, F1-score, and confusion matrix analysis.
    \item Fine-tune Further:  Based on evaluation results, fine-tune the model further, adjusting hyperparameters or modifying the architecture to improve performance.      \item \textbf {Inference and Prediction:}
    \item Make Predictions:  Use the trained model to predict brain tumor presence or absence in new MRI images.
    \item Post-processing:  Apply post-processing techniques (e.g., thresholding, morphological operations) to refine predictions and enhance accuracy.                         \item \textbf {Model Deployment:}
    \item Integration:  Integrate the trained model into a healthcare system or application for real-time brain tumor detection in MRI scans.
    \item Monitoring and Maintenance:  Continuously monitor and update the model to ensure optimal performance as new data becomes available or medical knowledge advances.
\end{itemize}
\subsubsection{Performance Requirements}
\begin{itemize}
    \item \textbf{Accuracy:}Achieve a minimum sensitivity of 90 percent and specificity of 95 percent in tumor detection.
    \item \textbf{Inference Time :}The system must complete tumor detection on an MRI scan in less than 3 seconds per image to support real-time clinical decision-making.
\end{itemize}
\subsubsection{Logical Database Requirements}
\begin{itemize}
    \item \textbf{Image Repository:} A logical database to store MRI images in a standardized format (e.g., DICOM) for easy retrieval and processing.
    \item \textbf{Model Metadata:} Metadata storage for pre-trained CNN models used in transfer learning, including model architecture, weights, and training history.
    \item \textbf{User Authentication:} A database to manage user accounts, roles, and permissions, ensuring secure access to patient data and system functionality.
    \item \textbf{Audit Trail:} A log database to record user actions, system events, and prediction results for monitoring and compliance purposes.
    \item \textbf{Configuration Settings:} Storage for user-defined settings, including pre-processing options and confidence thresholds for tumor classification.
\end{itemize}
\subsubsection{Design Constraints}
The design constraints of the Detecting of brain tumor in MRI’s using a convolutional neural networks with transfer
learning include:
\begin{itemize}
    \item \textbf{Data Availability and Quality:}Access to diverse and high-quality MRI datasets for training and validation is essential.
    \item \textbf{Computational Resources:} Sufficient computational power (CPU/GPU) for training and fine-tuning the CNN model.
    \item \textbf{Ethical Data Usage:} Adherence to ethical guidelines and patient privacy regulations when using medical data.
    \item \textbf{Integration with Healthcare Systems:} Compatibility with existing healthcare infrastructure for seamless integration into clinical workflows.
\end{itemize}

\subsubsection{Software System Quality Attributes}
The quality attributes of the Detecting of brain tumor in MRI’s using a convolutional neural networks with transfer
learning  include:
\begin{itemize}
    \item \textbf{Accuracy:} The system should have high accuracy and precision in detecting brain tumors in MRIs to minimize false positives and false negatives.
    \item \textbf{Performance:} The system should be optimized for high performance and speed to process MRIs efficiently, providing quick and timely results for medical professionals.
    \item \textbf{Scalability :} The system should be designed to handle a large volume of MRI scans, ensuring it can scale with increasing data and user demand without compromising performance.
    \item \textbf{Sensitivity and Specificity:} Sensitivity refers to the ability of the system to detect true positives. Specificity refers to the ability to avoid false positives. Both should be optimized for a reliable diagnosis.
\end{itemize}

\subsubsection{Object-Oriented Models}
 The Detecting of brain tumor in MRI’s using a convolutional neural networks with transfer learning can be represented using object-oriented models with the following key components:
\begin{itemize}
    \item \textbf{MRI Image:} Represents MRI image data and metadata for preprocessing and input to the CNN for tumor detection.
    \item \textbf{CNN Model:} Utilizes a pre-trained CNN with transfer learning for efficient and accurate feature extraction for tumor detection.
    \item \textbf{Tumor Detection Manager:} Implements the GCNN-based age and gender prediction model, taking the segmented teeth data as input and returning age and gender predictions.
    \item \textbf{Patient Record :}Maintains patient-specific data, recording MRI images and tumor detection outcomes for effective patient management.
    \item \textbf{MRI Scanner :}Simulates MRI scanning to generate MRI Image objects for testing and refining the tumor detection system.
\end{itemize}

\subsubsection{Appendices}
\textbf{1. Glossary}
\begin{itemize}
    \item This appendix can provide definitions of technical terms, acronyms, and abbreviations used throughout the document. It helps ensure that readers have a clear understanding of the terminology.
\end{itemize}
\textbf{2.Sample MRI Images}
\begin{itemize}
    \item Sample MRI images used for testing, validation, and development.
    \item Image annotations if available.

\end{itemize}
\textbf{3.Model Architecture Diagrams}
\begin{itemize}
    \item Diagrams illustrating the architecture of the CNN models used in the software, including details of layers and connections.
\end{itemize}
\textbf{4. References}
IEEE Std 830-1998, IEEE Recommended Practice for Software Requirements Specifications.






